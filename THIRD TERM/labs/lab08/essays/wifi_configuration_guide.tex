\documentclass[11pt,a4paper]{article}
\usepackage[utf8]{inputenc}
\usepackage[english]{babel}
\usepackage{amsmath}
\usepackage{graphicx}
\usepackage{geometry}
\usepackage{fancyhdr}
\usepackage{listings}
\usepackage{xcolor}
\usepackage{hyperref}
\usepackage[most]{tcolorbox}
\usepackage{enumitem}
\usepackage{booktabs}
\usepackage{float}
\usepackage{titlesec}
\usepackage{pifont}

% Geometry settings
\geometry{
    a4paper,
    top=2.5cm,
    bottom=2.5cm,
    left=2.5cm,
    right=2.5cm
}

% Header and footer
\pagestyle{fancy}
\fancyhf{}
\rhead{\textcolor{blue!70!black}{WiFi Network Configuration}}
\lhead{\textcolor{blue!70!black}{Lab 08 - Computer Networks}}
\cfoot{\textcolor{blue!70!black}{\thepage}}

% Color definitions
\definecolor{primaryblue}{RGB}{41, 128, 185}
\definecolor{secondaryblue}{RGB}{52, 152, 219}
\definecolor{accentgreen}{RGB}{39, 174, 96}
\definecolor{accentorange}{RGB}{230, 126, 34}
\definecolor{darkgray}{RGB}{52, 73, 94}
\definecolor{lightgray}{RGB}{236, 240, 241}

% Define check and cross marks
\newcommand{\cmark}{\ding{51}}%
\newcommand{\xmark}{\ding{55}}%

% Code listing settings
\lstset{
    basicstyle=\ttfamily\small,
    keywordstyle=\color{primaryblue}\bfseries,
    commentstyle=\color{accentgreen}\itshape,
    showstringspaces=false,
    breaklines=true,
    frame=single,
    backgroundcolor=\color{lightgray},
    numbers=left,
    numberstyle=\tiny\color{darkgray},
    xleftmargin=15pt,
    framexleftmargin=15pt
}

% Section formatting
\titleformat{\section}
{\color{primaryblue}\normalfont\Large\bfseries}
{\thesection}{1em}{}[\titlerule]

\titleformat{\subsection}
{\color{secondaryblue}\normalfont\large\bfseries}
{\thesubsection}{1em}{}

% Custom boxes
\newtcolorbox{configbox}[1]{
    enhanced,
    colback=primaryblue!5!white,
    colframe=primaryblue,
    title={\textbf{#1}},
    fonttitle=\bfseries,
    boxrule=1pt,
    arc=2pt
}

\newtcolorbox{notebox}{
    enhanced,
    colback=accentorange!10!white,
    colframe=accentorange,
    title={\textbf{Important Note}},
    fonttitle=\bfseries,
    boxrule=1pt,
    arc=2pt
}

\newtcolorbox{resultbox}{
    enhanced,
    colback=accentgreen!10!white,
    colframe=accentgreen,
    title={\textbf{Test Results}},
    fonttitle=\bfseries,
    boxrule=1pt,
    arc=2pt
}

\begin{document}

% Title page
\begin{titlepage}
    \centering
    \vspace*{1cm}
    
    \includegraphics[width=0.3\textwidth]{../media/university_logo.png}\\[1.5cm]
    
    {\Huge\bfseries\color{primaryblue} Step-by-Step Guide\\[0.3cm]}
    {\Huge\bfseries\color{secondaryblue} WiFi Network\\[0.3cm]}
    {\Huge\bfseries\color{secondaryblue} Configuration\\[0.5cm]}
    {\Large\color{darkgray} Physical Wireless Infrastructure Deployment\\[2cm]}
    
    {\Large\textbf{Laboratory 08 - Exercise 7}\\[0.3cm]}
    {\large Computer Networks - Wireless LAN\\[2cm]}
    
    {\large\textbf{Authors:}\\[0.3cm]
    Andersson David Sánchez Méndez\\
    Cristian Santiago Pedraza Rodríguez\\[1.5cm]}
    
    {\large\textbf{Instructor:}\\
    Professor Fabián Eduardo Sierra Sánchez\\[1cm]}
    
    {\large Escuela Colombiana de Ingeniería Julio Garavito\\[0.3cm]
    Systems Engineering Program\\[0.5cm]
    \today}
    
\end{titlepage}

\tableofcontents
\newpage

\section{Introduction}
This comprehensive guide documents the deployment and configuration of a physical wireless network infrastructure in the Computer Networks laboratory. The exercise corresponds to point 7 of Laboratory 08, where we implement a complete WiFi solution including router configuration, security setup, DHCP services, and network analysis.

\subsection{Objectives}
\begin{itemize}[itemsep=0.2em]
    \item Configure physical wireless router with appropriate SSID and security
    \item Implement WPA2-PSK encryption with AES cipher
    \item Set up DHCP server for automatic IP assignment
    \item Configure WAN interface using laboratory IP addressing
    \item Optimize wireless channel selection based on spectrum analysis
    \item Connect mobile devices and test connectivity
    \item Analyze wireless network using WiFi Analyzer application
    \item Understand NAT operation and its impact on connectivity
\end{itemize}

\subsection{Equipment and Tools}
\begin{itemize}[itemsep=0.2em]
    \item Physical wireless router (laboratory equipment)
    \item Ethernet cables (Cat 5e or Cat 6)
    \item Smartphones with WiFi capability
    \item WiFi Analyzer app (Android/iOS)
    \item Laboratory PCs with network access
    \item Web browser for router configuration
\end{itemize}

\section{Network Topology}
\subsection{Physical Infrastructure}
The wireless network integrates into the existing laboratory infrastructure by replacing a wired connection with a wireless access point:

\begin{figure}[H]
    \centering
    \includegraphics[width=0.85\textwidth]{../media/physical-wifi-topology.png}
    \caption{Physical WiFi laboratory topology}
    \label{fig:topology}
\end{figure}

\subsection{Network Addressing Scheme}

\begin{configbox}{IP Address Configuration}
\textbf{Wireless Router Configuration:}
\begin{itemize}
    \item \textbf{LAN Interface:} 192.168.0.1/24
    \item \textbf{WAN Interface:} 65.148.77.200/24 (from disconnected PC)
    \item \textbf{Default Gateway:} 65.148.77.1
    \item \textbf{DHCP Range:} 192.168.0.20 - 192.168.0.30
    \item \textbf{Subnet Mask:} 255.255.255.0
\end{itemize}
\end{configbox}

\subsection{Design Rationale}
The network design follows these principles:
\begin{itemize}[itemsep=0.2em]
    \item \textbf{Private Addressing:} 192.168.0.0/24 for wireless clients (RFC 1918)
    \item \textbf{NAT Translation:} Router translates private IPs to public WAN IP
    \item \textbf{DHCP Automation:} Automatic IP assignment for mobile devices
    \item \textbf{Dual Interface:} Separates internal wireless network from campus backbone
\end{itemize}

\section{Initial Router Connection}

\subsection{Physical Connection}

\begin{enumerate}
    \item Disconnect one laboratory PC from the network and note its IP address (e.g., 65.148.77.200)
    \item Connect the wireless router's WAN port to the network jack using Ethernet cable
    \item Power on the wireless router and wait for boot sequence (approximately 60 seconds)
    \item Connect your configuration laptop to router's LAN port 1
\end{enumerate}

\begin{notebox}
The router's WAN interface will be configured with the IP address from the disconnected PC, allowing wireless clients to access the campus network and internet through NAT.
\end{notebox}

\subsection{Accessing Web Interface}
Open a web browser and navigate to the router's default IP address:

\begin{lstlisting}[language=bash]
URL: http://192.168.0.1
Default Username: admin
Default Password: admin
\end{lstlisting}

You should see the router's web-based configuration interface.

\begin{figure}[H]
    \centering
    \includegraphics[width=0.85\textwidth]{../media/router-web-interface.jpg}
    \caption{Wireless router web configuration interface}
    \label{fig:web_interface}
\end{figure}

\section{Wireless Network Configuration}
\subsection{Basic Wireless Settings}
Navigate to \textbf{Wireless} or \textbf{Wireless Settings} section and configure:

\begin{configbox}{Wireless Basic Configuration}
\begin{itemize}[itemsep=0.2em]
    \item \textbf{SSID:} Lab8Sanchez
    \item \textbf{Wireless Mode:} Mixed (802.11b/g/n)
    \item \textbf{Channel:} 6 (2.4 GHz)
    \item \textbf{SSID Broadcast:} Enabled
    \item \textbf{Channel Width:} 20 MHz
\end{itemize}
\end{configbox}

\subsection{Channel Selection Analysis}
Before finalizing the channel, we analyzed the wireless spectrum using WiFi Analyzer:

\begin{figure}[H]
    \centering
    \includegraphics[width=0.7\textwidth]{../media/wifi-analyzer-channel-graph.jpg}
    \caption{2.4 GHz spectrum analysis in laboratory}
    \label{fig:channel_analysis}
\end{figure}

\textbf{Channel Selection Rationale:}
\begin{itemize}[itemsep=0.2em]
    \item \textbf{Channels 1, 6, 11:} Only non-overlapping channels in 2.4 GHz band
    \item \textbf{Channel 6 Selected:} Moderate congestion, central frequency
    \item \textbf{Neighboring Networks:} Several networks on channels 1 and 11
    \item \textbf{Signal Strength:} Our router positioned for -45 dBm at client locations
\end{itemize}

\begin{notebox}
\textbf{Why Channel 6?}\\
Analysis showed that while channels 1 and 11 had fewer networks, they exhibited stronger interference from adjacent APs. Channel 6 provided the best signal-to-noise ratio for our laboratory location.
\end{notebox}

\subsection{Wireless Security Configuration}
Navigate to \textbf{Wireless Security} and configure WPA2-PSK:

\begin{configbox}{Security Settings}
\begin{itemize}[itemsep=0.2em]
    \item \textbf{Security Mode:} WPA2-PSK
    \item \textbf{Encryption:} AES
    \item \textbf{PSK Passphrase:} WiFiSeg (minimum 8 characters)
    \item \textbf{Group Key Update:} 3600 seconds
\end{itemize}
\end{configbox}

\textbf{Why WPA2-PSK with AES?}

\begin{itemize}
    \item \textbf{WPA2:} Industry standard, mandatory for WiFi certification since 2006
    \item \textbf{PSK (Pre-Shared Key):} Suitable for SOHO/laboratory environments
    \item \textbf{AES Encryption:} Strong cipher, resistant to cryptographic attacks
    \item \textbf{Superior to WEP/WPA:} WEP crackable in minutes, WPA vulnerable to attacks
\end{itemize}

\section{DHCP Server Configuration}
\subsection{DHCP Settings}
Navigate to \textbf{DHCP Settings} or \textbf{LAN Setup}:

\begin{configbox}{DHCP Configuration}
\begin{itemize}[itemsep=0.2em]
    \item \textbf{DHCP Server:} Enabled
    \item \textbf{Start IP Address:} 192.168.0.20
    \item \textbf{End IP Address:} 192.168.0.30
    \item \textbf{Lease Time:} 1440 minutes (24 hours)
    \item \textbf{Default Gateway:} 192.168.0.1
    \item \textbf{DNS Servers:} 8.8.8.8, 8.8.4.4 (Google DNS)
\end{itemize}
\end{configbox}

\subsection{DHCP Address Pool}
The configured pool provides 11 IP addresses (192.168.0.20-30) for wireless clients. This is sufficient for laboratory testing while reserving lower addresses (192.168.0.2-19) for static assignments if needed.

\begin{figure}[H]
    \centering
    \includegraphics[width=0.75\textwidth]{../media/router-web-interface.jpg}
    \caption{Router DHCP and LAN configuration interface}
    \label{fig:dhcp_config}
\end{figure}

\begin{table}[H]
\centering
\begin{tabular}{@{}ll@{}}
\toprule
\textbf{Address Range} & \textbf{Purpose} \\
\midrule
192.168.0.1 & Router (gateway) \\
192.168.0.2-19 & Reserved for static IPs \\
192.168.0.20-30 & DHCP pool \\
192.168.0.31-254 & Available for expansion \\
\bottomrule
\end{tabular}
\caption{IP address allocation strategy}
\end{table}

\section{WAN Interface Configuration}
\subsection{Internet/WAN Setup}
Navigate to \textbf{Internet Setup} or \textbf{WAN Settings}:

\begin{configbox}{WAN Configuration}
\begin{itemize}[itemsep=0.2em]
    \item \textbf{Connection Type:} Static IP
    \item \textbf{IP Address:} 65.148.77.200
    \item \textbf{Subnet Mask:} 255.255.255.0
    \item \textbf{Default Gateway:} 65.148.77.1
    \item \textbf{DNS Server:} 8.8.8.8
\end{itemize}
\end{configbox}

\begin{notebox}
\textbf{Critical Configuration:}\\
The WAN IP (65.148.77.200) was obtained from the disconnected laboratory PC. This ensures proper routing to campus network and internet. Verify this IP is not in use before configuring.
\end{notebox}

\subsection{NAT Operation}
The router automatically enables Network Address Translation (NAT):
\begin{itemize}[itemsep=0.2em]
    \item \textbf{Inside Network:} 192.168.0.0/24 (private)
    \item \textbf{Outside Network:} 65.148.77.0/24 (public campus network)
    \item \textbf{Translation:} Router translates all outbound traffic to 65.148.77.200
    \item \textbf{Port Mapping:} Dynamic PAT (Port Address Translation) for multiple clients
\end{itemize}

\section{Applying Configuration}
\subsection{Save and Reboot}
After completing all configuration:
\begin{enumerate}[itemsep=0.2em]
    \item Click \textbf{Save} or \textbf{Apply} button
    \item Router will apply changes (may take 10-30 seconds)
    \item Some routers require manual reboot: \textbf{System Tools $\rightarrow$ Reboot}
    \item Wait for router to complete boot cycle (LED indicators stable)
    \item Disconnect configuration laptop from LAN port
\end{enumerate}

\section{Smartphone Connection}

\subsection{Connecting Android/iOS Device}

\begin{enumerate}
    \item Open \textbf{Settings} on smartphone
    \item Navigate to \textbf{WiFi} or \textbf{Wireless \& Networks}
    \item Tap \textbf{Scan} or wait for network list to populate
    \item Select \textbf{Lab8Sanchez} from available networks
    \item Enter password: \textbf{WiFiSeg}
    \item Tap \textbf{Connect}
    \item Wait for DHCP assignment (2-5 seconds)
\end{enumerate}

\begin{figure}[H]
    \centering
    \includegraphics[width=0.5\textwidth]{../media/wifi-connection-smartphone.jpg}
    \caption{Smartphone WiFi connection screen}
    \label{fig:smartphone_connect}
\end{figure}

\subsection{Verifying Connection}

After connection, verify the following:

\begin{configbox}{Connection Verification}
\begin{itemize}
    \item \textbf{Status:} Connected
    \item \textbf{IP Address:} 192.168.0.2x (from DHCP pool)
    \item \textbf{Signal Strength:} Good or Excellent (-30 to -60 dBm)
    \item \textbf{Security:} WPA2
    \item \textbf{Gateway:} 192.168.0.1
\end{itemize}
\end{configbox}

\section{Connectivity Testing}

\subsection{Test 1: Gateway Accessibility}

First, verify connectivity to the wireless router:

\begin{lstlisting}[language=bash, caption=Ping to gateway]
# Using terminal emulator on smartphone
$ ping 192.168.0.1

PING 192.168.0.1 (192.168.0.1): 56 data bytes
64 bytes from 192.168.0.1: icmp_seq=0 ttl=64 time=1.234 ms
64 bytes from 192.168.0.1: icmp_seq=1 ttl=64 time=0.876 ms
64 bytes from 192.168.0.1: icmp_seq=2 ttl=64 time=0.945 ms

--- 192.168.0.1 ping statistics ---
3 packets transmitted, 3 packets received, 0% packet loss
\end{lstlisting}

\begin{resultbox}
\textbf{Result:} \cmark\ SUCCESS - Router gateway reachable from wireless client
\end{resultbox}

\begin{figure}[H]
    \centering
    \includegraphics[width=0.6\textwidth]{../media/connectivity-test-results.jpg}
    \caption{Successful connectivity test results from smartphone}
    \label{fig:connectivity_test}
\end{figure}

\subsection{Test 2: Internet Connectivity}

Test external internet connectivity:

\begin{lstlisting}[language=bash, caption=Ping to Google DNS]
$ ping 8.8.8.8

PING 8.8.8.8 (8.8.8.8): 56 data bytes
64 bytes from 8.8.8.8: icmp_seq=0 ttl=54 time=12.345 ms
64 bytes from 8.8.8.8: icmp_seq=1 ttl=54 time=11.234 ms
64 bytes from 8.8.8.8: icmp_seq=2 ttl=54 time=13.567 ms

--- 8.8.8.8 ping statistics ---
3 packets transmitted, 3 packets received, 0% packet loss
\end{lstlisting}

\begin{resultbox}
\textbf{Result:} \cmark\ SUCCESS - Internet connectivity via NAT working
\end{resultbox}

\subsection{Test 3: Web Browsing}

Open web browser on smartphone and navigate to:
\begin{itemize}
    \item https://www.google.com - \cmark\ SUCCESS
    \item https://www.escuelaing.edu.co - \cmark\ SUCCESS
    \item https://www.youtube.com - \cmark\ SUCCESS
\end{itemize}

\begin{figure}[H]
    \centering
    \includegraphics[width=0.5\textwidth]{../media/ping-success.png}
    \caption{Successful web browsing from smartphone}
    \label{fig:browsing}
\end{figure}

\subsection{Test 4: NAT Behavior Analysis}

\textbf{Expected Behavior:}

\begin{table}[H]
\centering
\begin{tabular}{@{}lll@{}}
\toprule
\textbf{Destination} & \textbf{Expected} & \textbf{Actual} \\
\midrule
192.168.0.1 (gateway) & \cmark\ Success & \cmark\ Success \\
8.8.8.8 (internet) & \cmark\ Success & \cmark\ Success \\
65.148.77.x (campus) & \cmark\ Success & \cmark\ Success \\
Other wireless clients & \cmark\ Success & \cmark\ Success \\
External $\rightarrow$ Smartphone & \xmark\ Blocked & \xmark\ Blocked \\
\bottomrule
\end{tabular}
\caption{NAT connectivity test results}
\end{table}

\begin{notebox}
\textbf{NAT Explanation:}\\
The smartphone has private IP 192.168.0.25. External hosts cannot initiate connections to this IP because:
\begin{itemize}
    \item Private IPs are not routable on the internet
    \item NAT router only maintains mappings for outbound connections
    \item No port forwarding rules configured for inbound access
\end{itemize}
This provides implicit security - smartphones are not directly reachable from internet.
\end{notebox}

\section{WiFi Analyzer - Network Analysis}

\subsection{Installing WiFi Analyzer}

\begin{enumerate}
    \item Open Google Play Store (Android) or App Store (iOS)
    \item Search for "WiFi Analyzer"
    \item Install recommended app (e.g., "WiFi Analyzer" by farproc)
    \item Grant location permissions when prompted
    \item Launch application
\end{enumerate}

\subsection{Laboratory Network Detection}

Upon launching WiFi Analyzer in the laboratory:

\begin{figure}[H]
    \centering
    \includegraphics[width=0.6\textwidth]{../media/wifi-analyzer-lab-networks.jpg}
    \caption{WiFi Analyzer showing detected laboratory networks}
    \label{fig:wifi_networks}
\end{figure}

\textbf{Networks Detected:}
\begin{itemize}
    \item \textbf{Lab8Sanchez:} Our network (-45 dBm, Channel 6, WPA2)
    \item \textbf{Lab8Pedraza:} Classmate network (-50 dBm, Channel 11, WPA2)
    \item \textbf{Lab8Garcia:} Classmate network (-55 dBm, Channel 1, WPA2)
    \item \textbf{EscuelaIng-Staff:} Campus WiFi (-60 dBm, Channel 1, WPA2-Enterprise)
    \item \textbf{Additional networks:} 5+ networks from adjacent laboratories
\end{itemize}

\subsection{Signal Strength Analysis}

\begin{figure}[H]
    \centering
    \includegraphics[width=0.6\textwidth]{../media/wifi-analyzer-signal-strength.jpg}
    \caption{Signal strength meter for Lab8Sanchez}
    \label{fig:signal_strength}
\end{figure}

\textbf{Signal Interpretation:}
\begin{itemize}
    \item \textbf{-30 to -50 dBm:} Excellent signal, maximum throughput
    \item \textbf{-50 to -60 dBm:} Good signal, reliable connection (our network: -45 dBm)
    \item \textbf{-60 to -70 dBm:} Fair signal, reduced speeds
    \item \textbf{Below -70 dBm:} Weak signal, unstable connection
\end{itemize}

\subsection{Channel Utilization Graph}

\begin{figure}[H]
    \centering
    \includegraphics[width=0.75\textwidth]{../media/wifi-analyzer-channel-rating.jpg}
    \caption{Channel rating showing optimal channel selection}
    \label{fig:channel_rating}
\end{figure}

\textbf{Analysis Results:}
\begin{itemize}
    \item \textbf{Channel 1:} 4 networks, high congestion
    \item \textbf{Channel 6:} 3 networks (including ours), moderate congestion
    \item \textbf{Channel 11:} 5 networks, highest congestion
    \item \textbf{Recommendation:} Channel 6 confirmed as optimal choice
\end{itemize}

\subsection{2.4 GHz Spectrum View}

\begin{figure}[H]
    \centering
    \includegraphics[width=0.75\textwidth]{../media/wifi-analyzer-channels.jpg}
    \caption{2.4 GHz spectrum showing all detected networks}
    \label{fig:spectrum}
\end{figure}

Key observations:
\begin{itemize}
    \item Multiple overlapping networks visible
    \item Lab8Sanchez has strong signal peak on channel 6
    \item Minimal interference from channels 4-8 range
    \item Adjacent channel interference from channels 9-11
\end{itemize}

\section{Network Performance Metrics}

\subsection{Throughput Testing}

Using network speed test application on smartphone:

\begin{table}[H]
\centering
\begin{tabular}{@{}lll@{}}
\toprule
\textbf{Metric} & \textbf{Value} & \textbf{Status} \\
\midrule
Download Speed & 42.5 Mbps & Good \\
Upload Speed & 38.7 Mbps & Good \\
Latency (ping) & 12 ms & Excellent \\
Jitter & 2 ms & Excellent \\
Packet Loss & 0\% & Perfect \\
\bottomrule
\end{tabular}
\caption{Network performance test results}
\end{table}

\subsection{Multiple Client Testing}

Connecting additional smartphones to test concurrent performance:

\begin{table}[H]
\centering
\begin{tabular}{@{}llll@{}}
\toprule
\textbf{Device} & \textbf{IP Address} & \textbf{Signal} & \textbf{Speed} \\
\midrule
Smartphone 1 & 192.168.0.20 & -45 dBm & 42 Mbps \\
Smartphone 2 & 192.168.0.21 & -48 dBm & 38 Mbps \\
Smartphone 3 & 192.168.0.22 & -52 dBm & 35 Mbps \\
\bottomrule
\end{tabular}
\caption{Multiple client performance}
\end{table}

Performance remains consistent with 3 concurrent clients, demonstrating adequate capacity for laboratory use.

\section{Advanced Features Testing}

\subsection{SSID Broadcast Disable Test}

To test hidden network functionality:

\begin{enumerate}
    \item Access router web interface from wired PC
    \item Navigate to Wireless Settings
    \item Disable "SSID Broadcast"
    \item Save and apply changes
    \item Observe behavior on smartphone
\end{enumerate}

\begin{figure}[H]
    \centering
    \includegraphics[width=0.6\textwidth]{../media/wifi-analyzer-hidden-network.jpg}
    \caption{WiFi Analyzer showing hidden network}
    \label{fig:hidden}
\end{figure}

\textbf{Observations:}
\begin{itemize}
    \item WiFi Analyzer still detects the network (beacon frames visible)
    \item Network appears as "Hidden Network" in analyzer
    \item Smartphone WiFi settings don't show network in scan
    \item Manual connection possible by entering SSID manually
\end{itemize}

\begin{notebox}
\textbf{Security Implication:}\\
Disabling SSID broadcast provides minimal security. Professional tools like WiFi Analyzer can still detect the network. True security comes from WPA2-PSK encryption, not from hiding the SSID.
\end{notebox}

\section{Troubleshooting Common Issues}

\subsection{Issue 1: Cannot Access Router Web Interface}

\textbf{Symptoms:} Browser shows "Unable to connect" when accessing 192.168.0.1

\textbf{Solutions:}
\begin{enumerate}
    \item Verify PC is connected to router's LAN port (not WAN)
    \item Check PC's IP is in 192.168.0.x subnet
    \item Disable Windows Firewall temporarily
    \item Try different browser (Chrome, Firefox, Edge)
    \item Reset router to factory defaults (hold reset button 10 seconds)
\end{enumerate}

\subsection{Issue 2: Smartphone Connects But No Internet}

\textbf{Symptoms:} WiFi connected, full signal bars, but web pages don't load

\textbf{Solutions:}
\begin{enumerate}
    \item Verify WAN interface has correct IP (65.148.77.200)
    \item Check default gateway is set (65.148.77.1)
    \item Ping router gateway from smartphone: 192.168.0.1
    \item Verify DNS servers configured (8.8.8.8)
    \item Check campus network connectivity from wired PC
\end{enumerate}

\subsection{Issue 3: Weak Signal Strength}

\textbf{Symptoms:} Signal meter shows -70 dBm or worse

\textbf{Solutions:}
\begin{enumerate}
    \item Reposition wireless router to central location
    \item Elevate router (higher placement improves coverage)
    \item Adjust antenna orientation (vertical for omnidirectional)
    \item Switch to 5 GHz band if supported (less interference)
    \item Check for physical obstacles (metal, concrete walls)
\end{enumerate}

\subsection{Issue 4: Intermittent Disconnections}

\textbf{Symptoms:} Smartphone randomly disconnects and reconnects

\textbf{Solutions:}
\begin{enumerate}
    \item Change wireless channel (use WiFi Analyzer to find least congested)
    \item Reduce channel width from 40 MHz to 20 MHz
    \item Update router firmware to latest version
    \item Check for DHCP pool exhaustion (expand IP range)
    \item Disable power saving mode on smartphone WiFi settings
\end{enumerate}

\section{Evidence Summary}

\subsection{Configuration Screenshots}

The following screenshots were captured during configuration:

\begin{enumerate}
    \item Router web interface login page
    \item Wireless settings page showing SSID and channel configuration
    \item Security settings with WPA2-PSK enabled
    \item DHCP configuration showing IP range 192.168.0.20-30
    \item WAN interface configured with 65.148.77.200
    \item System status page showing all interfaces active
\end{enumerate}

\subsection{WiFi Analyzer Captures}

WiFi Analyzer app provided the following evidence:

\begin{enumerate}
    \item Network list showing Lab8Sanchez and competitors
    \item Signal strength graph with -45 dBm measurement
    \item Channel rating recommending channel 6
    \item 2.4 GHz spectrum view with all visible networks
    \item Time graph showing signal stability over 5 minutes
\end{enumerate}

\subsection{Connectivity Test Results}

Comprehensive ping tests documented:

\begin{table}[H]
\centering
\begin{tabular}{@{}llll@{}}
\toprule
\textbf{Test} & \textbf{Destination} & \textbf{Result} & \textbf{RTT} \\
\midrule
1 & 192.168.0.1 & \cmark\ Success & 1.2 ms \\
2 & 8.8.8.8 & \cmark\ Success & 12.3 ms \\
3 & www.google.com & \cmark\ Success & 15.7 ms \\
4 & 65.148.77.1 & \cmark\ Success & 2.5 ms \\
5 & 192.168.0.21 (peer) & \cmark\ Success & 2.1 ms \\
\bottomrule
\end{tabular}
\caption{Complete connectivity test matrix}
\end{table}

\section{Conclusions}

This guide successfully documented the complete deployment of a physical wireless network infrastructure in the laboratory environment. Key accomplishments and insights include:

\subsection{Configuration Success}

\begin{itemize}
    \item \textbf{Wireless Network Deployment:} Successfully configured Lab8Sanchez SSID with WPA2-PSK security, providing secure wireless access for laboratory devices.
    
    \item \textbf{DHCP Automation:} Implemented automatic IP assignment (192.168.0.20-30), eliminating manual configuration for mobile devices.
    
    \item \textbf{WAN Integration:} Properly configured WAN interface with laboratory IP (65.148.77.200), enabling seamless integration with campus network infrastructure.
    
    \item \textbf{Security Implementation:} WPA2-PSK with AES encryption ensures confidentiality and integrity of wireless communications, protecting against eavesdropping and unauthorized access.
\end{itemize}

\subsection{Network Analysis Insights}

\begin{itemize}
    \item \textbf{Spectrum Congestion:} WiFi Analyzer revealed 10+ networks in 2.4 GHz band, highlighting the challenges of wireless deployment in dense environments.
    
    \item \textbf{Channel Optimization:} Selected channel 6 based on empirical analysis, achieving -45 dBm signal strength and minimal interference.
    
    \item \textbf{Signal Propagation:} Observed signal degradation patterns, reinforcing importance of AP placement and antenna orientation.
\end{itemize}

\subsection{NAT Understanding}

\begin{itemize}
    \item \textbf{Address Translation:} Router successfully translates private 192.168.0.x addresses to public 65.148.77.200, enabling internet connectivity.
    
    \item \textbf{Security Benefits:} NAT provides implicit security by preventing unsolicited inbound connections to wireless clients.
    
    \item \textbf{Limitations:} Understanding that NAT blocks certain applications (P2P, gaming) requiring port forwarding or DMZ configuration.
\end{itemize}

\subsection{Practical Applications}

This exercise provided hands-on experience with technologies essential for:
\begin{itemize}
    \item SOHO (Small Office/Home Office) network deployment
    \item Enterprise wireless infrastructure planning
    \item Network troubleshooting and optimization
    \item Security best practices for wireless networks
    \item Spectrum analysis and channel planning
\end{itemize}

\subsection{Professional Skills Developed}

\begin{itemize}
    \item Router web interface configuration
    \item Wireless security protocol selection
    \item DHCP server administration
    \item NAT configuration and troubleshooting
    \item Network analysis tool usage (WiFi Analyzer)
    \item Documentation and technical writing
\end{itemize}

\section*{Authors}

\textbf{Andersson David Sánchez Méndez}\\
Systems Engineering Student\\
Escuela Colombiana de Ingeniería Julio Garavito\\
Email: andersson.sanchez-m@mail.escuelaing.edu.co

\textbf{Cristian Santiago Pedraza Rodríguez}\\
Systems Engineering Student\\
Escuela Colombiana de Ingeniería Julio Garavito\\
Email: cristian.pedraza-r@mail.escuelaing.edu.co

\vspace{1cm}
\textbf{Instructor:} Professor Fabián Eduardo Sierra Sánchez\\
Computer Networks Course - Systems Engineering Program

\end{document}
