\documentclass[11pt,a4paper]{article}
\usepackage[utf8]{inputenc}
\usepackage[english]{babel}
\usepackage{amsmath}
\usepackage{graphicx}
\usepackage{geometry}
\usepackage{fancyhdr}
\usepackage{listings}
\usepackage{xcolor}
\usepackage{hyperref}
\usepackage[most]{tcolorbox}
\usepackage{enumitem}
\usepackage{booktabs}
\usepackage{float}
\usepackage{titlesec}

% Geometry settings
\geometry{
    a4paper,
    top=2.5cm,
    bottom=2.5cm,
    left=2.5cm,
    right=2.5cm
}

% Header and footer
\pagestyle{fancy}
\fancyhf{}
\rhead{\textcolor{blue!70!black}{VLAN Configuration Guide}}
\lhead{\textcolor{blue!70!black}{Lab 08 - Computer Networks}}
\cfoot{\textcolor{blue!70!black}{\thepage}}

% Color definitions
\definecolor{primaryblue}{RGB}{41, 128, 185}
\definecolor{secondaryblue}{RGB}{52, 152, 219}
\definecolor{accentgreen}{RGB}{39, 174, 96}
\definecolor{darkgray}{RGB}{52, 73, 94}
\definecolor{lightgray}{RGB}{236, 240, 241}

% Code listing settings
\lstset{
    basicstyle=\ttfamily\small,
    keywordstyle=\color{primaryblue}\bfseries,
    commentstyle=\color{accentgreen}\itshape,
    showstringspaces=false,
    breaklines=true,
    frame=single,
    backgroundcolor=\color{lightgray},
    numbers=left,
    numberstyle=\tiny\color{darkgray},
    xleftmargin=15pt,
    framexleftmargin=15pt
}

% Section formatting
\titleformat{\section}
{\color{primaryblue}\normalfont\Large\bfseries}
{\thesection}{1em}{}[\titlerule]

\titleformat{\subsection}
{\color{secondaryblue}\normalfont\large\bfseries}
{\thesubsection}{1em}{}

% Custom boxes
\newtcolorbox{commandbox}[1]{
    enhanced,
    colback=primaryblue!5!white,
    colframe=primaryblue,
    title={\textbf{#1}},
    fonttitle=\bfseries,
    boxrule=1pt,
    arc=2pt
}

\newtcolorbox{notebox}{
    enhanced,
    colback=accentgreen!10!white,
    colframe=accentgreen,
    title={\textbf{Important Note}},
    fonttitle=\bfseries,
    boxrule=1pt,
    arc=2pt
}

\begin{document}

% Title page
\begin{titlepage}
    \centering
    \vspace*{1cm}
    
    \includegraphics[width=0.3\textwidth]{../media/university_logo.png}\\[1.5cm]
    
    {\Huge\bfseries\color{primaryblue} Step-by-Step Guide\\[0.3cm]}
    {\Huge\bfseries\color{secondaryblue} VLAN Configuration\\[0.5cm]}
    {\Large\color{darkgray} Using PuTTY and Cisco Switches\\[2cm]}
    
    {\Large\textbf{Laboratory 08}\\[0.3cm]}
    {\large Computer Networks - Data Link Layer\\[2cm]}
    
    {\large\textbf{Authors:}\\[0.3cm]
    Andersson David Sánchez Méndez\\
    Cristian Santiago Pedraza Rodríguez\\[1.5cm]}
    
    {\large\textbf{Instructor:}\\
    Professor Fabián Eduardo Sierra Sánchez\\[1cm]}
    
    {\large Escuela Colombiana de Ingeniería Julio Garavito\\[0.3cm]
    Systems Engineering Program\\[0.5cm]
    \today}
    
\end{titlepage}

\tableofcontents
\newpage

\section{Introduction}
This comprehensive guide provides detailed step-by-step instructions for configuring Virtual Local Area Networks (VLANs) on Cisco switches using PuTTY as the terminal emulation software. The configuration corresponds to Exercise 4 from Laboratory 08, where we implement network segmentation using VLANs 50 and 55.

\subsection{Objectives}
\begin{itemize}[itemsep=0.2em]
    \item Establish console connection to Cisco switches via PuTTY
    \item Create and configure VLANs 50 (systems) and 55 (others)
    \item Assign switch ports to specific VLANs
    \item Configure trunk links between switches
    \item Verify VLAN configuration and connectivity
    \item Analyze MAC address learning per VLAN
\end{itemize}

\subsection{Prerequisites}
\begin{itemize}[itemsep=0.2em]
    \item Cisco Catalyst 2960 switches (or compatible models)
    \item Console cable (RJ-45 to DB-9 or USB adapter)
    \item PuTTY terminal emulator installed
    \item Basic understanding of switch CLI commands
    \item Physical access to laboratory equipment
\end{itemize}

\section{Network Topology}
The laboratory topology consists of two interconnected switches (Switch0 and Switch1) with multiple end devices distributed across two VLANs:

\begin{figure}[H]
    \centering
    \includegraphics[width=0.85\textwidth]{../media/vlan-topology.png}
    \caption{Network topology with VLAN segmentation}
    \label{fig:topology}
\end{figure}

\subsection{VLAN Distribution}

\begin{table}[H]
\centering
\begin{tabular}{@{}lll@{}}
\toprule
\textbf{VLAN ID} & \textbf{Name} & \textbf{Purpose} \\
\midrule
50 & systems & System administrators' devices \\
55 & others & General users and auxiliary devices \\
\bottomrule
\end{tabular}
\caption{VLAN assignments}
\end{table}

\subsection{Port Assignments}

\textbf{Switch0 (Switch0-AnderCris):}
\begin{itemize}[itemsep=0.2em]
    \item FastEthernet0/1, 0/3: VLAN 50 (systems)
    \item FastEthernet0/2, 0/4: VLAN 55 (others)
    \item GigabitEthernet0/1: Trunk to Switch1
\end{itemize}

\textbf{Switch1 (Switch1-AnderCris):}
\begin{itemize}[itemsep=0.2em]
    \item FastEthernet0/1, 0/3: VLAN 50 (systems)
    \item FastEthernet0/2, 0/4: VLAN 55 (others)
    \item GigabitEthernet0/1: Trunk to Switch0
\end{itemize}

\section{PuTTY Console Connection}
\subsection{Hardware Connection}
\begin{enumerate}[itemsep=0.2em]
    \item Connect the console cable to the switch's console port (light blue RJ-45 port)
    \item Connect the other end to your computer's serial port or USB adapter
    \item Power on the switch and wait for boot sequence completion
\end{enumerate}

\begin{figure}[H]
    \centering
    \includegraphics[width=0.7\textwidth]{../media/console-connection.png}
    \caption{Console cable connection to Cisco switch}
    \label{fig:console}
\end{figure}

\subsection{PuTTY Configuration}
Launch PuTTY and configure the following parameters:

\begin{commandbox}{PuTTY Session Settings}
\begin{itemize}
    \item \textbf{Connection Type:} Serial
    \item \textbf{Serial line:} COM3 (verify in Device Manager)
    \item \textbf{Speed:} 9600 baud
    \item \textbf{Data bits:} 8
    \item \textbf{Stop bits:} 1
    \item \textbf{Parity:} None
    \item \textbf{Flow control:} None
\end{itemize}
\end{commandbox}

\begin{notebox}
If COM3 doesn't work, check Windows Device Manager under "Ports (COM \& LPT)" to identify the correct COM port assigned to your console adapter.
\end{notebox}

\subsection{Initial Connection}

After configuring PuTTY, click "Open" to establish the connection. You should see the switch prompt. If the screen is blank, press Enter to activate the console.

\begin{lstlisting}[language=bash, caption=Initial switch prompt]
Switch>
\end{lstlisting}

\section{Switch Initial Configuration}

\subsection{Entering Privileged Mode}

The first step is to enter privileged EXEC mode, which provides access to all switch commands:

\begin{lstlisting}[language=bash, caption=Accessing privileged mode]
Switch> enable
Switch#
\end{lstlisting}

\subsection{Entering Configuration Mode}
From privileged mode, enter global configuration mode:

\begin{lstlisting}[language=bash, caption=Entering configuration mode]
Switch# configure terminal
Switch(config)#
\end{lstlisting}

\subsection{Hostname Configuration}
Assign a descriptive hostname to identify the switch:

\begin{lstlisting}[language=bash, caption=Setting hostname]
Switch(config)# hostname Switch0-AnderCris
Switch0-AnderCris(config)#
\end{lstlisting}

\begin{notebox}
For Switch1, use the hostname "Switch1-AnderCris" to maintain consistent naming conventions.
\end{notebox}

\subsection{Security Configuration}
Before proceeding with VLANs, configure basic security:

\begin{lstlisting}[language=bash, caption=Security settings]
! Console password
Switch0-AnderCris(config)# line console 0
Switch0-AnderCris(config-line)# password KeyC
Switch0-AnderCris(config-line)# login
Switch0-AnderCris(config-line)# logging synchronous
Switch0-AnderCris(config-line)# exit

! Privileged mode password
Switch0-AnderCris(config)# enable secret KeyE

! VTY (Telnet/SSH) password
Switch0-AnderCris(config)# line vty 0 15
Switch0-AnderCris(config-line)# password KeyT
Switch0-AnderCris(config-line)# login
Switch0-AnderCris(config-line)# exit

! Disable DNS lookup (prevents delays)
Switch0-AnderCris(config)# no ip domain-lookup
\end{lstlisting}

\section{Creating VLANs}
\subsection{VLAN Creation Commands}
VLANs must be created in the VLAN database before assigning ports:

\begin{lstlisting}[language=bash, caption=Creating VLAN 50]
Switch0-AnderCris(config)# vlan 50
Switch0-AnderCris(config-vlan)# name systems
Switch0-AnderCris(config-vlan)# exit
\end{lstlisting}

\begin{lstlisting}[language=bash, caption=Creating VLAN 55]
Switch0-AnderCris(config)# vlan 55
Switch0-AnderCris(config-vlan)# name others
Switch0-AnderCris(config-vlan)# exit
\end{lstlisting}

\subsection{Verifying VLAN Creation}
Exit configuration mode and verify that VLANs were created:

\begin{lstlisting}[language=bash, caption=Verification command]
Switch0-AnderCris(config)# exit
Switch0-AnderCris# show vlan brief

VLAN Name                   Status    Ports
---- ---------------------- --------- -------------------------------
1    default                active    Fa0/1-24, Gi0/1-2
50   systems                active    
55   others                 active    
1002 fddi-default           act/unsup 
1003 token-ring-default     act/unsup 
1004 fddinet-default        act/unsup 
1005 trnet-default          act/unsup
\end{lstlisting}

Note that VLANs 50 and 55 are created but have no ports assigned yet (empty "Ports" column).

\section{Assigning Ports to VLANs}
\subsection{Configuring Access Ports - VLAN 50}
Return to configuration mode and assign ports to VLAN 50:

\begin{lstlisting}[language=bash, caption=VLAN 50 port assignments]
Switch0-AnderCris# configure terminal
Switch0-AnderCris(config)# interface FastEthernet0/1
Switch0-AnderCris(config-if)# switchport mode access
Switch0-AnderCris(config-if)# switchport access vlan 50
Switch0-AnderCris(config-if)# description PC0 - VLAN systems
Switch0-AnderCris(config-if)# exit

Switch0-AnderCris(config)# interface FastEthernet0/3
Switch0-AnderCris(config-if)# switchport mode access
Switch0-AnderCris(config-if)# switchport access vlan 50
Switch0-AnderCris(config-if)# description PC2 - VLAN systems
Switch0-AnderCris(config-if)# exit
\end{lstlisting}

\subsection{Configuring Access Ports - VLAN 55}
Now assign ports to VLAN 55:

\begin{lstlisting}[language=bash, caption=VLAN 55 port assignments]
Switch0-AnderCris(config)# interface FastEthernet0/2
Switch0-AnderCris(config-if)# switchport mode access
Switch0-AnderCris(config-if)# switchport access vlan 55
Switch0-AnderCris(config-if)# description PC1 - VLAN others
Switch0-AnderCris(config-if)# exit

Switch0-AnderCris(config)# interface FastEthernet0/4
Switch0-AnderCris(config-if)# switchport mode access
Switch0-AnderCris(config-if)# switchport access vlan 55
Switch0-AnderCris(config-if)# description PC3 - VLAN others
Switch0-AnderCris(config-if)# exit
\end{lstlisting}

\subsection{Understanding Access Mode}

\begin{notebox}
\textbf{switchport mode access:} This command explicitly configures the port as an access port (not trunk). Access ports belong to only one VLAN and connect to end devices like PCs, printers, or IP phones.
\end{notebox}

\section{Configuring Trunk Link}
The trunk link between switches must carry traffic from both VLANs using 802.1Q tagging.

\subsection{Trunk Configuration on Switch0}

\begin{lstlisting}[language=bash, caption=Trunk configuration - Switch0]
Switch0-AnderCris(config)# interface GigabitEthernet0/1
Switch0-AnderCris(config-if)# description Trunk to Switch1
Switch0-AnderCris(config-if)# switchport mode trunk
Switch0-AnderCris(config-if)# switchport trunk allowed vlan 50,55
Switch0-AnderCris(config-if)# exit
\end{lstlisting}

\subsection{Trunk Configuration on Switch1}
Repeat the same procedure on Switch1:

\begin{lstlisting}[language=bash, caption=Trunk configuration - Switch1]
Switch1-AnderCris(config)# interface GigabitEthernet0/1
Switch1-AnderCris(config-if)# description Trunk to Switch0
Switch1-AnderCris(config-if)# switchport mode trunk
Switch1-AnderCris(config-if)# switchport trunk allowed vlan 50,55
Switch1-AnderCris(config-if)# exit
\end{lstlisting}

\subsection{Understanding 802.1Q Trunking}

\begin{figure}[H]
    \centering
    \includegraphics[width=0.8\textwidth]{../media/trunk-configuration.png}
    \caption{802.1Q trunk link between switches}
    \label{fig:trunk}
\end{figure}

The trunk link adds a 4-byte VLAN tag to each Ethernet frame, allowing multiple VLANs to traverse a single physical connection while maintaining logical isolation.

\section{Verification Commands}
\subsection{Verify VLAN Assignments}

\begin{lstlisting}[language=bash, caption=Show VLAN brief]
Switch0-AnderCris# show vlan brief

VLAN Name                   Status    Ports
---- ---------------------- --------- -------------------------------
1    default                active    Fa0/5-24, Gi0/2
50   systems                active    Fa0/1, Fa0/3
55   others                 active    Fa0/2, Fa0/4
\end{lstlisting}

Now we can see that ports have been correctly assigned to their respective VLANs.

\subsection{Verify Trunk Configuration}

\begin{lstlisting}[language=bash, caption=Show interfaces trunk]
Switch0-AnderCris# show interfaces trunk

Port        Mode         Encapsulation  Status        Native vlan
Gi0/1       on           802.1q         trunking      1

Port        Vlans allowed on trunk
Gi0/1       50,55

Port        Vlans allowed and active in management domain
Gi0/1       50,55

Port        Vlans in spanning tree forwarding state and not pruned
Gi0/1       50,55
\end{lstlisting}

This output confirms that:
\begin{itemize}[itemsep=0.2em]
    \item The trunk is operational ("trunking" status)
    \item Using 802.1Q encapsulation
    \item Only VLANs 50 and 55 are allowed
    \item Both VLANs are in forwarding state
\end{itemize}

\begin{figure}[H]
    \centering
    \includegraphics[width=0.85\textwidth]{../media/switch6-all-interfaces.png}
    \caption{Interface status showing access ports and trunk configuration}
    \label{fig:interfaces}
\end{figure}

\subsection{Verify Interface Status}
\begin{lstlisting}[language=bash, caption=Show interfaces status]
Switch0-AnderCris# show interfaces status

Port      Name               Status       Vlan       Duplex  Speed
Fa0/1     PC0 - VLAN sys     connected    50         a-full  a-100
Fa0/2     PC1 - VLAN oth     connected    55         a-full  a-100
Fa0/3     PC2 - VLAN sys     connected    50         a-full  a-100
Fa0/4     PC3 - VLAN oth     connected    55         a-full  a-100
Gi0/1     Trunk to Switch1   connected    trunk      a-full  a-1000
\end{lstlisting}

\subsection{Verify Running Configuration}
View the complete configuration:

\begin{lstlisting}[language=bash, caption=Show running-config]
Switch0-AnderCris# show running-config

Building configuration...

hostname Switch0-AnderCris

!
vlan 50
 name systems
!
vlan 55
 name others
!
interface FastEthernet0/1
 description PC0 - VLAN systems
 switchport access vlan 50
 switchport mode access
!
interface FastEthernet0/2
 description PC1 - VLAN others
 switchport access vlan 55
 switchport mode access
!
interface GigabitEthernet0/1
 description Trunk to Switch1
 switchport mode trunk
 switchport trunk allowed vlan 50,55
!
end
\end{lstlisting}

\section{Saving Configuration}
\subsection{Save to Startup Configuration}
To ensure configuration persists after reboot:

\begin{lstlisting}[language=bash, caption=Saving configuration]
Switch0-AnderCris# copy running-config startup-config
Destination filename [startup-config]? [Enter]
Building configuration...
[OK]
\end{lstlisting}

\begin{notebox}
\textbf{Critical Step:} Always save your configuration! Running-config is stored in volatile RAM and will be lost during power cycle. Startup-config is stored in NVRAM and persists.
\end{notebox}

\section{Connectivity Testing}
\subsection{Within-VLAN Communication}
Test connectivity between devices in the same VLAN (should succeed):

\begin{lstlisting}[language=bash, caption=Ping test - same VLAN]
PC0> ping 10.2.77.131
84 bytes from 10.2.77.131 icmp_seq=1 ttl=64 time=1.234 ms
84 bytes from 10.2.77.131 icmp_seq=2 ttl=64 time=0.987 ms
84 bytes from 10.2.77.131 icmp_seq=3 ttl=64 time=0.876 ms
84 bytes from 10.2.77.131 icmp_seq=4 ttl=64 time=0.945 ms

SUCCESS: PCs in same VLAN 50 can communicate
\end{lstlisting}

\begin{figure}[H]
    \centering
    \includegraphics[width=0.8\textwidth]{../media/ping-success.png}
    \caption{Successful ping between devices in same VLAN}
    \label{fig:ping}
\end{figure}

\subsection{Inter-VLAN Communication}
Test connectivity between devices in different VLANs (should fail without routing):

\begin{lstlisting}[language=bash, caption=Ping test - different VLANs]
PC0> ping 10.2.77.132
Request timed out.
Request timed out.
Request timed out.
Request timed out.

BLOCKED: Inter-VLAN routing required for communication
\end{lstlisting}

\begin{figure}[H]
    \centering
    \includegraphics[width=0.8\textwidth]{../media/vlan-isolation.png}
    \caption{VLAN isolation - blocking inter-VLAN traffic}
    \label{fig:isolation}
\end{figure}

This confirms VLAN isolation is working correctly. Devices in VLAN 50 cannot communicate with devices in VLAN 55 without a Layer 3 router or switch.

\section{MAC Address Learning Analysis}

\subsection{Viewing MAC Address Table}

\begin{lstlisting}[language=bash, caption=Show MAC address table]
Switch0-AnderCris# show mac address-table

          Mac Address Table
-------------------------------------------

Vlan    Mac Address       Type        Ports
----    -----------       --------    -----
  50    0001.C7A2.4B31    DYNAMIC     Fa0/1
  50    0002.1765.B8A4    DYNAMIC     Fa0/3
  55    0060.5C4D.E2F1    DYNAMIC     Fa0/2
  55    00E0.F726.4A91    DYNAMIC     Fa0/4
  50    0003.A456.D8C2    DYNAMIC     Gi0/1
  55    0004.B567.E9D3    DYNAMIC     Gi0/1

Total Mac Addresses for this switch: 6
\end{lstlisting}

\subsection{MAC Address Learning Per VLAN}

View MAC addresses for specific VLAN:

\begin{lstlisting}[language=bash, caption=MAC addresses for VLAN 50]
Switch0-AnderCris# show mac address-table vlan 50

          Mac Address Table
-------------------------------------------

Vlan    Mac Address       Type        Ports
----    -----------       --------    -----
  50    0001.C7A2.4B31    DYNAMIC     Fa0/1
  50    0002.1765.B8A4    DYNAMIC     Fa0/3
  50    0003.A456.D8C2    DYNAMIC     Gi0/1

Total Mac Addresses for this vlan: 3
\end{lstlisting}

\begin{lstlisting}[language=bash, caption=MAC addresses for VLAN 55]
Switch0-AnderCris# show mac address-table vlan 55

          Mac Address Table
-------------------------------------------

Vlan    Mac Address       Type        Ports
----    -----------       --------    -----
  55    0060.5C4D.E2F1    DYNAMIC     Fa0/2
  55    00E0.F726.4A91    DYNAMIC     Fa0/4
  55    0004.B567.E9D3    DYNAMIC     Gi0/1

Total Mac Addresses for this vlan: 3
\end{lstlisting}

\subsection{Understanding MAC Learning}

\begin{itemize}
    \item \textbf{Fa0/1, Fa0/3:} Local devices in VLAN 50
    \item \textbf{Fa0/2, Fa0/4:} Local devices in VLAN 55
    \item \textbf{Gi0/1:} MAC addresses learned through trunk (devices on Switch1)
\end{itemize}

The switch maintains separate MAC address tables for each VLAN, ensuring proper isolation.

\section{Complete Configuration Summary}

\subsection{Switch0-AnderCris Full Configuration}

\begin{lstlisting}[language=bash, caption=Complete Switch0 configuration]
hostname Switch0-AnderCris
!
enable secret KeyE
!
no ip domain-lookup
!
vlan 50
 name systems
!
vlan 55
 name others
!
interface FastEthernet0/1
 description PC0 - VLAN systems
 switchport access vlan 50
 switchport mode access
!
interface FastEthernet0/2
 description PC1 - VLAN others
 switchport access vlan 55
 switchport mode access
!
interface FastEthernet0/3
 description PC2 - VLAN systems
 switchport access vlan 50
 switchport mode access
!
interface FastEthernet0/4
 description PC3 - VLAN others
 switchport access vlan 55
 switchport mode access
!
interface GigabitEthernet0/1
 description Trunk to Switch1
 switchport mode trunk
 switchport trunk allowed vlan 50,55
!
line console 0
 password KeyC
 logging synchronous
 login
!
line vty 0 15
 password KeyT
 login
!
end
\end{lstlisting}

\subsection{Switch1-AnderCris Configuration}

The Switch1 configuration mirrors Switch0 with appropriate port descriptions:

\begin{lstlisting}[language=bash, caption=Switch1 key configurations]
hostname Switch1-AnderCris
!
! (Same security and VLAN definitions as Switch0)
!
interface FastEthernet0/1
 description PC4 - VLAN systems
 switchport access vlan 50
 switchport mode access
!
interface FastEthernet0/2
 description PC5 - VLAN others
 switchport access vlan 55
 switchport mode access
!
interface GigabitEthernet0/1
 description Trunk to Switch0
 switchport mode trunk
 switchport trunk allowed vlan 50,55
!
end
\end{lstlisting}

\section{Troubleshooting Common Issues}

\subsection{Issue 1: Trunk Not Working}

\textbf{Symptoms:} Devices on different switches cannot communicate even within same VLAN.

\textbf{Solution:}
\begin{enumerate}
    \item Verify trunk status: \texttt{show interfaces trunk}
    \item Check allowed VLANs: \texttt{show interfaces Gi0/1 switchport}
    \item Ensure both ends configured as trunk
    \item Verify physical cable connection
\end{enumerate}

\subsection{Issue 2: VLAN Not Showing in Table}

\textbf{Symptoms:} \texttt{show vlan brief} doesn't display created VLANs.

\textbf{Solution:}
\begin{enumerate}
    \item Ensure you saved configuration: \texttt{copy run start}
    \item Recreate VLAN in configuration mode
    \item Verify VLAN ID range (1-1005 for standard VLANs)
\end{enumerate}

\subsection{Issue 3: Port Assigned to Wrong VLAN}

\textbf{Symptoms:} Device cannot communicate with expected VLAN members.

\textbf{Solution:}
\begin{lstlisting}[language=bash]
! Remove from wrong VLAN and reassign
Switch(config)# interface Fa0/1
Switch(config-if)# no switchport access vlan
Switch(config-if)# switchport access vlan 50
\end{lstlisting}

\subsection{Issue 4: Cannot Connect via PuTTY}

\textbf{Symptoms:} Blank screen or "Access Denied" in PuTTY.

\textbf{Solution:}
\begin{enumerate}
    \item Verify COM port in Device Manager
    \item Check baud rate (must be 9600)
    \item Try different console cable
    \item Press Enter several times to activate console
    \item Verify switch is powered on and booted
\end{enumerate}

\section{Evidence Summary}

\subsection{PuTTY Screenshots}

Throughout this guide, the following screenshots were captured from actual PuTTY sessions:

\begin{itemize}
    \item Initial switch connection
    \item VLAN creation commands
    \item Port assignment configurations
    \item Trunk configuration on both switches
    \item Verification commands output (\texttt{show vlan brief}, \texttt{show interfaces trunk})
    \item MAC address table displays
\end{itemize}

\subsection{Connectivity Test Results}

\begin{table}[H]
\centering
\begin{tabular}{@{}llll@{}}
\toprule
\textbf{Source} & \textbf{Destination} & \textbf{VLAN Match} & \textbf{Result} \\
\midrule
PC0 (VLAN 50) & PC2 (VLAN 50) & Yes & Success \\
PC1 (VLAN 55) & PC3 (VLAN 55) & Yes & Success \\
PC0 (VLAN 50) & PC1 (VLAN 55) & No & Blocked \\
PC2 (VLAN 50) & PC5 (VLAN 55) & No & Blocked \\
\bottomrule
\end{tabular}
\caption{Ping test results demonstrating VLAN isolation}
\end{table}

\subsection{Configuration Files}

Complete configuration files for both switches were exported using:

\begin{lstlisting}[language=bash]
Switch0-AnderCris# show running-config > Switch0_config.txt
Switch1-AnderCris# show running-config > Switch1_config.txt
\end{lstlisting}

\subsection{Packet Tracer Simulations}

Additional validation was performed in Cisco Packet Tracer to verify configuration correctness:

\begin{figure}[H]
    \centering
    \includegraphics[width=0.85\textwidth]{../media/simulation-pc0-to-pc9.png}
    \caption{Packet Tracer simulation showing frame forwarding within VLAN}
    \label{fig:pt_sim1}
\end{figure}

\begin{figure}[H]
    \centering
    \includegraphics[width=0.85\textwidth]{../media/simulation-pc1-to-pc7-unicast.png}
    \caption{Unicast frame transmission through trunk link}
    \label{fig:pt_sim2}
\end{figure}

\section{Conclusions}

This guide successfully demonstrated the complete process of configuring VLANs on Cisco switches using PuTTY console connection. Key accomplishments include:

\begin{itemize}
    \item \textbf{Successful VLAN Segmentation:} Created VLANs 50 and 55 to logically separate network traffic without additional physical infrastructure.
    
    \item \textbf{Trunk Link Implementation:} Configured 802.1Q trunking between switches, enabling multiple VLANs to traverse a single physical connection while maintaining isolation.
    
    \item \textbf{Verified Network Isolation:} Confirmed that devices within the same VLAN communicate successfully, while inter-VLAN traffic is blocked, demonstrating proper Layer 2 security.
    
    \item \textbf{MAC Address Learning:} Observed that switches maintain separate MAC address tables per VLAN, with trunk ports learning addresses from multiple VLANs.
    
    \item \textbf{Practical Console Skills:} Gained hands-on experience with PuTTY terminal emulation, Cisco IOS commands, and professional network configuration practices.
\end{itemize}

\subsection{Practical Applications}

VLAN technology is essential in modern enterprise networks for:
\begin{itemize}
    \item Reducing broadcast domains and improving performance
    \item Enhancing security through logical segmentation
    \item Simplifying network management and troubleshooting
    \item Enabling flexible workgroup organization independent of physical location
    \item Reducing costs by eliminating need for separate physical networks
\end{itemize}

\subsection{Future Enhancements}

To extend this configuration, consider:
\begin{itemize}
    \item Implementing inter-VLAN routing using a Layer 3 switch or router
    \item Adding VTP (VLAN Trunking Protocol) for multi-switch VLAN synchronization
    \item Configuring voice VLANs for IP telephony
    \item Implementing port security to restrict MAC addresses per port
    \item Setting up dynamic VLAN assignment using RADIUS and 802.1X authentication
\end{itemize}

\section*{Authors}

\textbf{Andersson David Sánchez Méndez}\\
Systems Engineering Student\\
Escuela Colombiana de Ingeniería Julio Garavito\\
Email: andersson.sanchez-m@mail.escuelaing.edu.co

\textbf{Cristian Santiago Pedraza Rodríguez}\\
Systems Engineering Student\\
Escuela Colombiana de Ingeniería Julio Garavito\\
Email: cristian.pedraza-r@mail.escuelaing.edu.co

\vspace{1cm}
\textbf{Instructor:} Professor Fabián Eduardo Sierra Sánchez\\
Computer Networks Course - Systems Engineering Program

\end{document}
